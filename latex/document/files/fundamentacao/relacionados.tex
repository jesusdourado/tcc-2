%!TEX root=../../sbc-template.tex

A malária ainda é um problema mundial e o padrão ouro para o diagnóstico da  doença utiliza a análise de imagens microscópicas do sangue do paciente, um processo considerado lento e que depende de especialistas. Quando feito tardiamente compromete drasticamente o desfecho do tratamento, essencial para uma boa recuperação do paciente \cite{Dalrymple:2015}. Assim, soluções que empreguem Visão Computacional para mitigar esse problema são desejadas, pois além de acelerarem o diagnóstico podem ajudar com \emph{insights} e prognóstico acerca tanto da doença como o estado do paciente \cite{Ewerton:2019}.

Observando o estado da arte sobre \emph{Deep Learning} para o diagnóstico de Malária, ressalta-se o trabalho de Guimarães e Guedes (2022) \nocite{Felipe:2022}. Os autores utilizaram uma base de dados contendo $\num{27588}$ imagens de células do sangue humano oriunda de um pré-processamento em imagens de lâminas de exame de sangue. Os modelos de CNNs abordados pelos autores que realizam a tarefa de classificação foram: VGG-16, InceptionV3, ResNet-50 e EfficientNetB0,  cada um com o treinamento utilizando 250 épocas. Foram obtidas ótimas métricas de desempenho, destacando-se a Precisão e F-Score \cite{Felipe:2022}.

Um trabalho correlato de importante menção foi realizado por Yang e outros, os quais fizeram uso da mesma base de dados considerada no escopo desse trabalho. Os autores consideraram duas tarefas de Aprendizado de Máquina para abordar o problema:  a localização dos protozoários e sua posterior classificação. Nessa última tarefa, utilizaram os modelos VGG-19, ResNet-50, AlexNet e uma arquitetura própria de CNN. Como os modelos mencionados eram voltados apenas para  a tarefa de classificação, as imagens originais sofreram um pré-processamento para destacar as regiões candidatas por meio de um algoritmo de janelamento. Por fim, ao aferir o desempenho, ressaltam-se métricas satisfatórias no que tange à classificação, com especial destaque para AUC, Acurácia e Precisão, inclusive para o modelo  proposto \cite{Yang:2020}.

Vale ressaltar que, dos trabalhos citados, a tarefa principal dos modelos de \emph{Deep Learning} utilizando CNNs foi a classificação, ou seja, as imagens de lâminas de sangue para o exame da malária não poderiam ser usadas tal como foram capturadas, sendo necessária uma etapa de pré-processamento, quer feita por um especialista ou por um algoritmo de localização. No trabalho de Guimarães e Guedes, a base de dados possuía todas as células do sangue e protozoários devidamente individualizadas \cite{Felipe:2022}. Em relação ao trabalho de Yang e outros, um algoritmo para destacar as células candidatas precisou ser aplicado à base de dados para se extrair as imagens que, posteriormente, foram fornecidas aos modelos de classificação \cite{Yang:2020}. A fim de contornar essas limitações observadas na literatura, o presente trabalho utilizou as imagens microscópicas originais obtidas nos exames como entrada para um modelo de \emph{Deep Learning} para detecção de objetos \emph{single-shot} da Família YOLO, o qual localiza e classifica todos os  protozoários na imagem original em uma única passada da imagem de entrada pela rede, sem qualquer demanda ou ônus de pré-processamento.