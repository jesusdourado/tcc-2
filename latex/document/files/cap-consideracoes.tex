%!TEX root=../sbc-template.tex

\chapter{Considerações Finais e Trabalhos Futuros} \label{cap:considera}

O presente trabalho de conclusão de curso mostrou a viabilidade dos modelos de detecção de objetos baseados em \emph{Deep Learning} da Família YOLO no tocante à tarefa de detecção de protozoários causadoras de malária em imagens de lâminas de sangue oriundas de microscópio. Esses resultados foram observados a partir de experimentos computacionais com modelos YOLOv5 e YOLOv7 em uma base de dados realística da literatura contendo objetos pequenos e uniformemente distribuídos na imagem, o que a tornou especialmente desafiadora. As entradas dos modelos não necessitaram de pré-processamento computacional ou de intervenção humana para sua preparação para os modelos, o que se mostrou um avanço no estado da arte frente à outras contribuições identificadas na literatura. Os resultados evidenciaram a YOLOv7 \emph{X} como tendo melhor desempenho nos experimentos realizados, mas não foi possível assegurar a superioridade desse modelo frente aos demais, sendo necessários mais experimentos e também a ponderação quanto aos recursos computacionais disponível no âmbito de utilização da solução. Nesse sentido, ressalta-se que os resultados obtidos não se propõem em hipótese alguma a substituir especialistas humanos na tarefa, sendo essencial aprofundar as análises para verificar tal viabilidade.

Em trabalhos futuros almeja-se avaliar outras versões de modelos da Família YOLO recém propostos na literatura, tais como a YOLOv8 \cite{Jocher:YOLOv8}. Além disso, é importante validar o desempenho dos modelos em mais exemplos oriundos de mais localidades, especialmente da Região Amazônica onde tal doença é endêmica. Para tanto, encoraja-se pesquisadores de outras áreas a contribuir com bases de dados para esta tarefa de Visão Computacional.