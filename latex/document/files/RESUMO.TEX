\ \ \\[2cm]

\noindent Este trabalho aborda a doença da malária no âmbito global, nacional e regional, pontuando as características do modo de transmissão, vetores e sintomas, além de mencionar o diagnóstico precoce utilizando imagens microscópicas para, então, aplicar o tratamento eficaz. Neste contexto, o trabalho apresenta a seguinte solução: por meio de \emph{Deep Learning} aplicada à Visão Computacional, foi desenvolvido um modelo, baseado na arquitetura YOLO, capaz de facilitar o processo de diagnóstico microscópio da malária realizado por profissionais de saúde. Os resultados obtidos foram finalmente avaliados, levando em consideração quais modelos alcançaram melhores métricas de desempenho em relação ao custo computacional de espaço e tempo, mostrando que a solução proposta é viável para o problema postulado.

\ \ \newline

\noindent \textbf{Palavras-Chave}: Deep Learning; YOLO; Malária.
