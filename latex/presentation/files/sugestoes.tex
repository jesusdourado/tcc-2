%!TEX root=../main.tex
\begin{frame}{Estudo de Caso: Pavimentações no Brasil}
    \begin{block}{Questão Motriz}
    Como o melhor modelo identificado se comporta em um cenário brasileiro?
    \end{block}
    \pause
    \begin{itemize}
    \item \emph{Road Traversing Knowledge} -- \alert{RTK \emph{dataset}} \cite{Rateke2019}
    \item Anotado originalmente para segmentação
    \item Imagens com dimensões $352 \times 288$ \emph{pixels}
    \end{itemize}
\end{frame}

\begin{frame}{Estudo de Caso: RTK \emph{dataset}}
   \begin{table}[h!]
\begin{center}
\caption{Descrição estatística do pré-processamento do RTK \emph{dataset}.} \label{tab:boundingboxes}
\begin{footnotesize}
\begin{tabular}{cccccccccc}
\toprule
        & \multicolumn{3}{c}{\textbf{Comprimento}} & & \multicolumn{3}{c}{\textbf{Largura}} & & \textbf{Exemplos} \\
        \cmidrule{2-4} \cmidrule{6-8} \cmidrule{10-10}
      &  \textbf{Média} & \textbf{Máx} & \textbf{Mín} & &    \textbf{Média} & \textbf{Máx} & \textbf{Mín} & & \textbf{Quantidade}\\
\midrule
\textbf{Classe D00} & $43.27 \pm 26.06$ & 98 & 6 & & $33.61 \pm 21.17$ & 87 & 3 & & 36\\
\textbf{Classe D10} & $18.76 \pm 21.00$ & 107 & 2 & & $42.24 \pm 47.70$ & 345 & 5 & & 390\\
\textbf{Classe D40} & $13.92 \pm 10.11$ & 62 & 3 & & $34.52 \pm 17.38$ & 93 & 4 & & 105\\
\bottomrule
\end{tabular}
\end{footnotesize}
\end{center}
\end{table}
\end{frame}


\begin{frame}{Estudo de Caso: Exemplos}
    \begin{figure}[h!]
    	\centering
    	\hfill {\includegraphics[width=0.32\linewidth]{./img/cracks_1.jpg}}
    	\subfloat{\includegraphics[width=0.32\linewidth]{./img/cracks_2.jpg}}
    	\subfloat{\includegraphics[width=0.32\linewidth]{./img/pothole_6.jpg}}
    	\caption{Exemplos do RTK \emph{dataset} com rótulos de classes mapeados.} \label{img:rtk}
    \end{figure}
\end{frame}

\begin{frame}{Estudo de Caso: Metodologia}
\begin{itemize}
    \item Transferência de aprendizado utilizando a arquitetura YOLOv5 com a Configuração 4
    \item Validação cruzada \emph{holdout}
    \begin{itemize}
    \item Treinamento: $70\%$
    \item Validação: $10\%$
    \item Teste: $20\%$
\end{itemize}
\ \ \newline
\item Regularização com o uso de \alert{\emph{early stopping}}
\item \alert{Métricas}: Precisão, Revocação, $F_1$-\emph{Score} e mAP
\end{itemize}
\end{frame}


\begin{frame}{Estudo de Caso: Resultados}
\begin{table}[h!]
\begin{center}
\caption{Desempenho da rede na previsão de falhas em pavimentações asfálticas no Brasil.} \label{tab:yoloBrasil}
\begin{tabular}{ccccc}
	\toprule
	  & \textbf{Precisão} & \textbf{Revocação} & $\mathbf{F_1}$\textbf{-Score} & \textbf{mAP}$\mathbf{@0.5}$\\
	\midrule
\textbf{Todas as Classes} & \SI{43,8}{\percent} & \SI{51,0}{\percent} & \SI{47,1}{\percent} & \SI{45,6}{\percent}\\
\textbf{Classe D00} & \SI{18,7}{\percent} & \SI{23,4}{\percent} & \SI{20,8}{\percent} & \SI{11,5}{\percent}\\
\textbf{Classe D10} & \SI{42,6}{\percent} & \SI{45,0}{\percent} & \SI{43,8}{\percent} & \SI{40,9}{\percent}\\
\textbf{Classe D40} & \SI{70,2}{\percent} & \SI{84,6}{\percent} & \SI{76,7}{\percent} & \SI{84,3}{\percent}\\
\bottomrule
\end{tabular}
\end{center}
\end{table}

\end{frame}

\begin{frame}{Estudo de Caso: Resultados}
\begin{figure}[h!]
	\centering
	\hfill \subfloat[Exemplo 1 -- Desejado]{\includegraphics[width=0.45\linewidth]{./img/pothole_6.jpg}} 	\hfill
	\subfloat[Exemplo 1 -- Previsto]{\includegraphics[width=0.45\linewidth]{./img/detected_6.jpg}} \hfill
    \caption{Exemplos de previsão do modelo no cenário brasileiro.}
\end{figure}
\end{frame}

\begin{frame}{Estudo de Caso: Resultados}
\begin{figure}[h!]
	\centering
	\hfill\subfloat[Exemplo 2 -- Desejado]
	{\includegraphics[width=0.45\linewidth]{./img/pothole_5.jpg}}\hfill
	\subfloat[Exemplo 2 -- Previsto]
	{\includegraphics[width=0.45\linewidth]{./img/detected_5.jpg}}\hfill
	\caption{Exemplos de previsão do modelo no cenário brasileiro.}
	
\end{figure}
\end{frame}

\begin{frame}{Estudo de Caso: Análise dos Resultados}
\begin{itemize}
    \item Transferência Negativa de Aprendizado: decréscimo percentual do mAP@0.5 de $\SI{14,25}{\percent}$
    \ \ \newline
    \item Tamanho das \emph{bounding boxes}, resolução das imagens, distância entre domínio de origem e de aplicação, etc.
    \ \ \newline
    \item Necessidade de elaborar bases de dados representativas para o Brasil
\end{itemize}

\end{frame}
