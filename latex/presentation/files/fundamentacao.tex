%!TEX root=../main.tex

\begin{frame}{Fundamentação Teórica}
\begin{itemize}
    \item \alert{Rede Neural Artificial (RNA)}: composta por neurônios conectados numa rede que realizando cálculos sobre entradas e emitindo saídas
    %\ \ \newline
    \item \alert{Rede Neural Convolucional (CNN)}: utilizam filtros convolucionais que trabalham com entradas multidimensionais
    %\ \ \newline
    \item  \alert{Rede Neural Convolucional Regional (R-CNN)}: voltadas para detecção de objetos
    \begin{itemize}
        \item Janelamento
        \item \emph{Selective Search}
        \item Outras CNNs
    \end{itemize}
    \item CNNs e R-CNNs estão em constante evolução
\end{itemize}
\end{frame}

\begin{frame}{Fundamentação Teórica: YOLO}
\begin{itemize}
    \item \alert{You Only Look Once}: abordagem \emph{single-shot} \cite{Redmon:YOLOoriginal}
    \ \ \newline
    \item O modelo mostrou-se mais rápido e eficiente que os vigentes até agora
    \ \ \newline
    \item Aplicações em tempo real, até 120 FPS
\end{itemize}
\end{frame}

\begin{frame}{Fundamentação Teórica: YOLO}

\begin{block}{YOLOv5 \cite{Yolov5}}
\begin{itemize}
    \item Melhora na acurácia, velocidade de treinamento e de inferência, além de uma redução na quantidade de pesos
    \item Desenvolvida nativamente com \emph{framework Pytorch}, visando maior suporte e processo de \emph{deploy} simplificado 
\end{itemize}

\end{block}
\begin{block}{YOLOv7 \cite{yolov7}}
\begin{itemize}
    \item Aprimoramento dos modelos ao incorporar um processo de reparametrização
    \item Estratégia de dimensionamento em escala para outras quantidades de parâmetros
\end{itemize}
\end{block}

\end{frame}
