%!TEX root=../main.tex

\begin{frame}{Motivação e Contextualização}
\begin{itemize}
    \item A malária é uma doença causada pelo protozoário do gênero \emph{Plasmodium}, transmitida pelo mosquito do gênero \emph{Anopheles} \cite{OMS:Malaria2019}
    \ \ \newline
    \item A enfermidade é um problema grave de saúde, sendo uma das principais doenças letais em regiões tropicais \cite{OMS:Malaria2019}
    \ \ \newline
    \item Segundo a OMS, foram registrados 228 milhões de casos e 405 mil mortes em âmbito global, a grande maioria ocorreu na África 
     \ \ \newline
    \item O Brasil, no anos de 2020 e 2021, registrou 141 mil e 135 mil casos, respectivamente \cite{Boletim:Malaria2022}
\end{itemize}
\end{frame}

\begin{frame}{Motivação e Contextualização}
    \begin{itemize}
        \item O diagnóstico precoce é importante para o tratamento adequado \cite{Berzosa:MalariaDiagnostico}
        \ \ \newline
        \item  Existem 2 tipos de diagnósticos: \cite{OMS:Malaria2019}
        \begin{itemize}
            \item Análise microscópica de lâminas de sangue
            \item Testes Rápidos de Diagnósticos (RDTs)
        \end{itemize}
        \ \ \newline
        \item Automatizar a análise das lâminas para o protozoário presentes no sangue
        \begin{itemize}
        \item Diminuir a quantidade de especialistas necessários e acelerar o diagnóstico
        \end{itemize}
    \end{itemize}
\end{frame}

\begin{frame}{Objetivos}
        \begin{block}{Objetivo Geral}
            Avaliar modelos de \emph{Deep Learning} para detecção de malária em imagens microscópicas
        \end{block}
        \begin{block}{Objetivos Específicos}
       \begin{enumerate}
    \item Identificar e preparar uma base de dados da literatura no domínio da detecção de malária
    \item Elaborar e conduzir um estudo de caso experimental para treinamento e avaliação comparativa das arquiteturas de CNNs para detecção de objetos
    \item Avaliar e analisar os resultados obtidos dos modelos
\end{enumerate}
\end{block}
\end{frame}
 
%\begin{frame}{Publicação}
%        \begin{block}{Artigo Completo Aceito para Publicação}
%        \begin{itemize}
%            \item CARVALHO, R. B.; GUEDES, E. B.; FIGUEIREDO, C. M. S.; %\textbf{Detecção Inteligente de Falhas em Pavimentações Asfálticas com Redes %Neurais Convolucionais Regionais}. Trilha Principal. III Workshop Brasileiro %de Cidades Inteligentes (WBCI) do XLII Congresso da Sociedade Brasileira de %Computação (CSBC 2022), 12 pp.
%            \item Evento será realizado entre os dias 31/07/2022 e %05/08/2022 em Niterói, RJ.
%        \end{itemize}
%        \end{block}
%\end{frame}

